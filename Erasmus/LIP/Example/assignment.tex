% Options for packages loaded elsewhere
% Options for packages loaded elsewhere
\PassOptionsToPackage{unicode}{hyperref}
\PassOptionsToPackage{hyphens}{url}
\PassOptionsToPackage{dvipsnames,svgnames,x11names}{xcolor}
%
\documentclass[
  letterpaper,
  DIV=11,
  numbers=noendperiod]{scrartcl}
\usepackage{xcolor}
\usepackage{amsmath,amssymb}
\setcounter{secnumdepth}{-\maxdimen} % remove section numbering
\usepackage{iftex}
\ifPDFTeX
  \usepackage[T1]{fontenc}
  \usepackage[utf8]{inputenc}
  \usepackage{textcomp} % provide euro and other symbols
\else % if luatex or xetex
  \usepackage{unicode-math} % this also loads fontspec
  \defaultfontfeatures{Scale=MatchLowercase}
  \defaultfontfeatures[\rmfamily]{Ligatures=TeX,Scale=1}
\fi
\usepackage{lmodern}
\ifPDFTeX\else
  % xetex/luatex font selection
  \setmainfont[]{Latin Modern Roman}
  \setsansfont[]{Latin Modern Roman}
\fi
% Use upquote if available, for straight quotes in verbatim environments
\IfFileExists{upquote.sty}{\usepackage{upquote}}{}
\IfFileExists{microtype.sty}{% use microtype if available
  \usepackage[]{microtype}
  \UseMicrotypeSet[protrusion]{basicmath} % disable protrusion for tt fonts
}{}
\makeatletter
\@ifundefined{KOMAClassName}{% if non-KOMA class
  \IfFileExists{parskip.sty}{%
    \usepackage{parskip}
  }{% else
    \setlength{\parindent}{0pt}
    \setlength{\parskip}{6pt plus 2pt minus 1pt}}
}{% if KOMA class
  \KOMAoptions{parskip=half}}
\makeatother
% Make \paragraph and \subparagraph free-standing
\makeatletter
\ifx\paragraph\undefined\else
  \let\oldparagraph\paragraph
  \renewcommand{\paragraph}{
    \@ifstar
      \xxxParagraphStar
      \xxxParagraphNoStar
  }
  \newcommand{\xxxParagraphStar}[1]{\oldparagraph*{#1}\mbox{}}
  \newcommand{\xxxParagraphNoStar}[1]{\oldparagraph{#1}\mbox{}}
\fi
\ifx\subparagraph\undefined\else
  \let\oldsubparagraph\subparagraph
  \renewcommand{\subparagraph}{
    \@ifstar
      \xxxSubParagraphStar
      \xxxSubParagraphNoStar
  }
  \newcommand{\xxxSubParagraphStar}[1]{\oldsubparagraph*{#1}\mbox{}}
  \newcommand{\xxxSubParagraphNoStar}[1]{\oldsubparagraph{#1}\mbox{}}
\fi
\makeatother

\usepackage{color}
\usepackage{fancyvrb}
\newcommand{\VerbBar}{|}
\newcommand{\VERB}{\Verb[commandchars=\\\{\}]}
\DefineVerbatimEnvironment{Highlighting}{Verbatim}{commandchars=\\\{\}}
% Add ',fontsize=\small' for more characters per line
\usepackage{framed}
\definecolor{shadecolor}{RGB}{241,243,245}
\newenvironment{Shaded}{\begin{snugshade}}{\end{snugshade}}
\newcommand{\AlertTok}[1]{\textcolor[rgb]{0.68,0.00,0.00}{#1}}
\newcommand{\AnnotationTok}[1]{\textcolor[rgb]{0.37,0.37,0.37}{#1}}
\newcommand{\AttributeTok}[1]{\textcolor[rgb]{0.40,0.45,0.13}{#1}}
\newcommand{\BaseNTok}[1]{\textcolor[rgb]{0.68,0.00,0.00}{#1}}
\newcommand{\BuiltInTok}[1]{\textcolor[rgb]{0.00,0.23,0.31}{#1}}
\newcommand{\CharTok}[1]{\textcolor[rgb]{0.13,0.47,0.30}{#1}}
\newcommand{\CommentTok}[1]{\textcolor[rgb]{0.37,0.37,0.37}{#1}}
\newcommand{\CommentVarTok}[1]{\textcolor[rgb]{0.37,0.37,0.37}{\textit{#1}}}
\newcommand{\ConstantTok}[1]{\textcolor[rgb]{0.56,0.35,0.01}{#1}}
\newcommand{\ControlFlowTok}[1]{\textcolor[rgb]{0.00,0.23,0.31}{\textbf{#1}}}
\newcommand{\DataTypeTok}[1]{\textcolor[rgb]{0.68,0.00,0.00}{#1}}
\newcommand{\DecValTok}[1]{\textcolor[rgb]{0.68,0.00,0.00}{#1}}
\newcommand{\DocumentationTok}[1]{\textcolor[rgb]{0.37,0.37,0.37}{\textit{#1}}}
\newcommand{\ErrorTok}[1]{\textcolor[rgb]{0.68,0.00,0.00}{#1}}
\newcommand{\ExtensionTok}[1]{\textcolor[rgb]{0.00,0.23,0.31}{#1}}
\newcommand{\FloatTok}[1]{\textcolor[rgb]{0.68,0.00,0.00}{#1}}
\newcommand{\FunctionTok}[1]{\textcolor[rgb]{0.28,0.35,0.67}{#1}}
\newcommand{\ImportTok}[1]{\textcolor[rgb]{0.00,0.46,0.62}{#1}}
\newcommand{\InformationTok}[1]{\textcolor[rgb]{0.37,0.37,0.37}{#1}}
\newcommand{\KeywordTok}[1]{\textcolor[rgb]{0.00,0.23,0.31}{\textbf{#1}}}
\newcommand{\NormalTok}[1]{\textcolor[rgb]{0.00,0.23,0.31}{#1}}
\newcommand{\OperatorTok}[1]{\textcolor[rgb]{0.37,0.37,0.37}{#1}}
\newcommand{\OtherTok}[1]{\textcolor[rgb]{0.00,0.23,0.31}{#1}}
\newcommand{\PreprocessorTok}[1]{\textcolor[rgb]{0.68,0.00,0.00}{#1}}
\newcommand{\RegionMarkerTok}[1]{\textcolor[rgb]{0.00,0.23,0.31}{#1}}
\newcommand{\SpecialCharTok}[1]{\textcolor[rgb]{0.37,0.37,0.37}{#1}}
\newcommand{\SpecialStringTok}[1]{\textcolor[rgb]{0.13,0.47,0.30}{#1}}
\newcommand{\StringTok}[1]{\textcolor[rgb]{0.13,0.47,0.30}{#1}}
\newcommand{\VariableTok}[1]{\textcolor[rgb]{0.07,0.07,0.07}{#1}}
\newcommand{\VerbatimStringTok}[1]{\textcolor[rgb]{0.13,0.47,0.30}{#1}}
\newcommand{\WarningTok}[1]{\textcolor[rgb]{0.37,0.37,0.37}{\textit{#1}}}

\usepackage{longtable,booktabs,array}
\usepackage{calc} % for calculating minipage widths
% Correct order of tables after \paragraph or \subparagraph
\usepackage{etoolbox}
\makeatletter
\patchcmd\longtable{\par}{\if@noskipsec\mbox{}\fi\par}{}{}
\makeatother
% Allow footnotes in longtable head/foot
\IfFileExists{footnotehyper.sty}{\usepackage{footnotehyper}}{\usepackage{footnote}}
\makesavenoteenv{longtable}
\usepackage{graphicx}
\makeatletter
\newsavebox\pandoc@box
\newcommand*\pandocbounded[1]{% scales image to fit in text height/width
  \sbox\pandoc@box{#1}%
  \Gscale@div\@tempa{\textheight}{\dimexpr\ht\pandoc@box+\dp\pandoc@box\relax}%
  \Gscale@div\@tempb{\linewidth}{\wd\pandoc@box}%
  \ifdim\@tempb\p@<\@tempa\p@\let\@tempa\@tempb\fi% select the smaller of both
  \ifdim\@tempa\p@<\p@\scalebox{\@tempa}{\usebox\pandoc@box}%
  \else\usebox{\pandoc@box}%
  \fi%
}
% Set default figure placement to htbp
\def\fps@figure{htbp}
\makeatother





\setlength{\emergencystretch}{3em} % prevent overfull lines

\providecommand{\tightlist}{%
  \setlength{\itemsep}{0pt}\setlength{\parskip}{0pt}}



 


\KOMAoption{captions}{tableheading}
\makeatletter
\@ifpackageloaded{caption}{}{\usepackage{caption}}
\AtBeginDocument{%
\ifdefined\contentsname
  \renewcommand*\contentsname{Table of contents}
\else
  \newcommand\contentsname{Table of contents}
\fi
\ifdefined\listfigurename
  \renewcommand*\listfigurename{List of Figures}
\else
  \newcommand\listfigurename{List of Figures}
\fi
\ifdefined\listtablename
  \renewcommand*\listtablename{List of Tables}
\else
  \newcommand\listtablename{List of Tables}
\fi
\ifdefined\figurename
  \renewcommand*\figurename{Figure}
\else
  \newcommand\figurename{Figure}
\fi
\ifdefined\tablename
  \renewcommand*\tablename{Table}
\else
  \newcommand\tablename{Table}
\fi
}
\@ifpackageloaded{float}{}{\usepackage{float}}
\floatstyle{ruled}
\@ifundefined{c@chapter}{\newfloat{codelisting}{h}{lop}}{\newfloat{codelisting}{h}{lop}[chapter]}
\floatname{codelisting}{Listing}
\newcommand*\listoflistings{\listof{codelisting}{List of Listings}}
\makeatother
\makeatletter
\makeatother
\makeatletter
\@ifpackageloaded{caption}{}{\usepackage{caption}}
\@ifpackageloaded{subcaption}{}{\usepackage{subcaption}}
\makeatother
\usepackage{bookmark}
\IfFileExists{xurl.sty}{\usepackage{xurl}}{} % add URL line breaks if available
\urlstyle{same}
\hypersetup{
  pdftitle={Sensitivity Analysis and Revised Simplex},
  pdfauthor={nesko25},
  colorlinks=true,
  linkcolor={blue},
  filecolor={Maroon},
  citecolor={Blue},
  urlcolor={Blue},
  pdfcreator={LaTeX via pandoc}}


\title{Sensitivity Analysis and Revised Simplex}
\author{nesko25}
\date{}
\begin{document}
\maketitle


\section{1 - Mathematical Formulation}\label{mathematical-formulation}

Given the data, the factory-planning problem model will look like this:

\[
\begin{aligned}
\max \  & 19x_1 + 13x_2 + 12x_3 + 17x_4 \\
s.t. \ & 3x_1 + 2x_2 + x_3 + 2x_4 \le 225 \\
& x_1 + x_2 + x_3 + x_4 \le 117 \\
& 4x_1 + 3x_2 + 3x_3 + 4x_4 \le 420 \\
& x_1, x_2, x_3, x_4 \ge 0
\end{aligned}
\]

We can implement it in python and bring to the equational standard form
to get the initial tableau:

\begin{Shaded}
\begin{Highlighting}[]
\ImportTok{import}\NormalTok{ numpy }\ImportTok{as}\NormalTok{ np}
\ImportTok{from}\NormalTok{ fractions }\ImportTok{import}\NormalTok{ Fraction}
\ImportTok{from}\NormalTok{ util }\ImportTok{import}\NormalTok{ tableau}

\NormalTok{n }\OperatorTok{=} \DecValTok{4}
\NormalTok{m }\OperatorTok{=} \DecValTok{3}

\CommentTok{\# model in standard form}
\NormalTok{A }\OperatorTok{=}\NormalTok{ np.array([}
\NormalTok{    [}\DecValTok{3}\NormalTok{, }\DecValTok{2}\NormalTok{, }\DecValTok{1}\NormalTok{, }\DecValTok{2}\NormalTok{],}
\NormalTok{    [}\DecValTok{1}\NormalTok{, }\DecValTok{1}\NormalTok{, }\DecValTok{1}\NormalTok{, }\DecValTok{1}\NormalTok{],}
\NormalTok{    [}\DecValTok{4}\NormalTok{, }\DecValTok{3}\NormalTok{, }\DecValTok{3}\NormalTok{, }\DecValTok{4}\NormalTok{]}
\NormalTok{])}
\NormalTok{b }\OperatorTok{=}\NormalTok{ np.array([}\DecValTok{225}\NormalTok{, }\DecValTok{117}\NormalTok{, }\DecValTok{420}\NormalTok{])}
\NormalTok{c }\OperatorTok{=}\NormalTok{ np.array([}\DecValTok{19}\NormalTok{, }\DecValTok{13}\NormalTok{, }\DecValTok{12}\NormalTok{, }\DecValTok{17}\NormalTok{])}

\CommentTok{\# equational standard form}
\NormalTok{A }\OperatorTok{=}\NormalTok{ np.hstack([A, np.identity(m)])}
\NormalTok{c }\OperatorTok{=}\NormalTok{ np.hstack([c, np.zeros(m)])}

\CommentTok{\# inital tableau}
\NormalTok{T }\OperatorTok{=}\NormalTok{ np.column\_stack([}
\NormalTok{    np.vstack([A, c]),}
\NormalTok{    np.append(np.zeros(m), [}\DecValTok{1}\NormalTok{]), }\CommentTok{\# {-}z}
\NormalTok{    np.append(b, [}\DecValTok{0}\NormalTok{]) }\CommentTok{\# b}
\NormalTok{])}

\CommentTok{\# make it pretty}
\NormalTok{T }\OperatorTok{=}\NormalTok{ np.array(T, dtype }\OperatorTok{=} \BuiltInTok{object}\NormalTok{)}
\ControlFlowTok{for}\NormalTok{ j }\KeywordTok{in} \BuiltInTok{range}\NormalTok{(m }\OperatorTok{+} \DecValTok{1}\NormalTok{):}
    \ControlFlowTok{for}\NormalTok{ i }\KeywordTok{in} \BuiltInTok{range}\NormalTok{(n }\OperatorTok{+}\NormalTok{ m }\OperatorTok{+} \DecValTok{2}\NormalTok{):}
\NormalTok{        T[j][i] }\OperatorTok{=}\NormalTok{ Fraction(T[j][i])}

\NormalTok{tableau(T)}
\end{Highlighting}
\end{Shaded}

\begin{verbatim}
|-------+-------+-------+-------+-------+-------+-------+-------+-------+
|    x1 |    x2 |    x3 |    x4 |    x5 |    x6 |    x7 |    -z |     b |
|-------+-------+-------+-------+-------+-------+-------+-------+-------+
|     3 |     2 |     1 |     2 |     1 |     0 |     0 |     0 |   225 |
|     1 |     1 |     1 |     1 |     0 |     1 |     0 |     0 |   117 |
|     4 |     3 |     3 |     4 |     0 |     0 |     1 |     0 |   420 |
|-------+-------+-------+-------+-------+-------+-------+-------+-------+
|    19 |    13 |    12 |    17 |     0 |     0 |     0 |     1 |     0 |
|-------+-------+-------+-------+-------+-------+-------+-------+-------+
\end{verbatim}

\newpage

\section{2 - Optimal Tableau}\label{optimal-tableau}

\subsection{Using Original Simplex}\label{using-original-simplex}

Choose variable to bring to the basis

\begin{Shaded}
\begin{Highlighting}[]
\KeywordTok{def}\NormalTok{ ratio\_test():}
\NormalTok{    enter }\OperatorTok{=}\NormalTok{ np.argmax(T[}\OperatorTok{{-}}\DecValTok{1}\NormalTok{][:n }\OperatorTok{+}\NormalTok{ m])}
    \ControlFlowTok{if}\NormalTok{ T[}\OperatorTok{{-}}\DecValTok{1}\NormalTok{][enter] }\OperatorTok{\textless{}=} \DecValTok{0}\NormalTok{:}
        \ControlFlowTok{return}\NormalTok{ [}\VariableTok{None}\NormalTok{, }\VariableTok{None}\NormalTok{] }\CommentTok{\# optimal}

\NormalTok{    ratios }\OperatorTok{=}\NormalTok{ [}
\NormalTok{        T[j][}\OperatorTok{{-}}\DecValTok{1}\NormalTok{] }\OperatorTok{/}\NormalTok{ T[j][enter]}
        \ControlFlowTok{if}\NormalTok{ T[j][enter] }\OperatorTok{\textgreater{}} \DecValTok{0}
        \ControlFlowTok{else} \BuiltInTok{float}\NormalTok{(}\StringTok{\textquotesingle{}inf\textquotesingle{}}\NormalTok{)}
        \ControlFlowTok{for}\NormalTok{ j }\KeywordTok{in} \BuiltInTok{range}\NormalTok{(m)}
\NormalTok{    ]}

\NormalTok{    leave }\OperatorTok{=}\NormalTok{ np.argmin(ratios)}
    \ControlFlowTok{if}\NormalTok{ ratios[leave] }\OperatorTok{==} \BuiltInTok{float}\NormalTok{(}\StringTok{\textquotesingle{}inf\textquotesingle{}}\NormalTok{):}
        \ControlFlowTok{return}\NormalTok{ [}\VariableTok{None}\NormalTok{, }\VariableTok{None}\NormalTok{] }\CommentTok{\# unbounded}

    \ControlFlowTok{return}\NormalTok{ [enter, leave]}

\NormalTok{enter, leave }\OperatorTok{=}\NormalTok{ ratio\_test()}
\BuiltInTok{print}\NormalTok{(}\SpecialStringTok{f"pivot: (}\SpecialCharTok{\{}\NormalTok{enter}\SpecialCharTok{\}}\SpecialStringTok{, }\SpecialCharTok{\{}\NormalTok{leave}\SpecialCharTok{\}}\SpecialStringTok{)"}\NormalTok{)}
\end{Highlighting}
\end{Shaded}

\begin{verbatim}
pivot: (0, 0)
\end{verbatim}

So according to the ratio\_test \(x_1\) should enter and \(x_5\) should
leave

\begin{Shaded}
\begin{Highlighting}[]
\KeywordTok{def}\NormalTok{ update\_tableau(enter, leave):}
\NormalTok{    T[leave] }\OperatorTok{/=}\NormalTok{ T[leave][enter]}
    \ControlFlowTok{for}\NormalTok{ j }\KeywordTok{in} \BuiltInTok{range}\NormalTok{(m }\OperatorTok{+} \DecValTok{1}\NormalTok{):}
        \ControlFlowTok{if}\NormalTok{ j }\OperatorTok{!=}\NormalTok{ leave:}
\NormalTok{            T[j] }\OperatorTok{{-}=}\NormalTok{ T[leave] }\OperatorTok{*}\NormalTok{ T[j][enter]}

\NormalTok{update\_tableau(enter, leave)}
\NormalTok{tableau(T)}
\end{Highlighting}
\end{Shaded}

\begin{verbatim}
|-------+-------+-------+-------+-------+-------+-------+-------+-------+
|    x1 |    x2 |    x3 |    x4 |    x5 |    x6 |    x7 |    -z |     b |
|-------+-------+-------+-------+-------+-------+-------+-------+-------+
|     1 |   2/3 |   1/3 |   2/3 |   1/3 |     0 |     0 |     0 |    75 |
|     0 |   1/3 |   2/3 |   1/3 |  -1/3 |     1 |     0 |     0 |    42 |
|     0 |   1/3 |   5/3 |   4/3 |  -4/3 |     0 |     1 |     0 |   120 |
|-------+-------+-------+-------+-------+-------+-------+-------+-------+
|     0 |   1/3 |  17/3 |  13/3 | -19/3 |     0 |     0 |     1 | -1425 |
|-------+-------+-------+-------+-------+-------+-------+-------+-------+
\end{verbatim}

And we repeat:

\begin{Shaded}
\begin{Highlighting}[]
\KeywordTok{def}\NormalTok{ pivot():}
\NormalTok{    enter, leave }\OperatorTok{=}\NormalTok{ ratio\_test()}
\NormalTok{    update\_tableau(enter, leave)}
\NormalTok{    tableau(T)}

\NormalTok{pivot()}
\end{Highlighting}
\end{Shaded}

\begin{verbatim}
|-------+-------+-------+-------+-------+-------+-------+-------+-------+
|    x1 |    x2 |    x3 |    x4 |    x5 |    x6 |    x7 |    -z |     b |
|-------+-------+-------+-------+-------+-------+-------+-------+-------+
|     1 |   1/2 |     0 |   1/2 |   1/2 |  -1/2 |     0 |     0 |    54 |
|     0 |   1/2 |     1 |   1/2 |  -1/2 |   3/2 |     0 |     0 |    63 |
|     0 |  -1/2 |     0 |   1/2 |  -1/2 |  -5/2 |     1 |     0 |    15 |
|-------+-------+-------+-------+-------+-------+-------+-------+-------+
|     0 |  -5/2 |     0 |   3/2 |  -7/2 | -17/2 |     0 |     1 | -1782 |
|-------+-------+-------+-------+-------+-------+-------+-------+-------+
\end{verbatim}

\begin{Shaded}
\begin{Highlighting}[]
\NormalTok{pivot()}
\end{Highlighting}
\end{Shaded}

\begin{verbatim}
|-------+-------+-------+-------+-------+-------+-------+-------+-------+
|    x1 |    x2 |    x3 |    x4 |    x5 |    x6 |    x7 |    -z |     b |
|-------+-------+-------+-------+-------+-------+-------+-------+-------+
|     1 |     1 |     0 |     0 |     1 |     2 |    -1 |     0 |    39 |
|     0 |     1 |     1 |     0 |     0 |     4 |    -1 |     0 |    48 |
|     0 |    -1 |     0 |     1 |    -1 |    -5 |     2 |     0 |    30 |
|-------+-------+-------+-------+-------+-------+-------+-------+-------+
|     0 |    -1 |     0 |     0 |    -2 |    -1 |    -3 |     1 | -1827 |
|-------+-------+-------+-------+-------+-------+-------+-------+-------+
\end{verbatim}

And we've reached optimality as there are no more positive reduced
costs. The solution agrees with the statement in the exercise that in
the optimal solution \(x_1\), \(x_3\) and \(x_4\) will be in basis.

\subsection{Using Revised Simplex}\label{using-revised-simplex}

\begin{Shaded}
\begin{Highlighting}[]
\NormalTok{B }\OperatorTok{=}\NormalTok{ [}\DecValTok{0}\NormalTok{, }\DecValTok{2}\NormalTok{, }\DecValTok{3}\NormalTok{] }\CommentTok{\# final basis}
\NormalTok{N }\OperatorTok{=}\NormalTok{ [}\DecValTok{1}\NormalTok{, }\DecValTok{4}\NormalTok{, }\DecValTok{5}\NormalTok{, }\DecValTok{6}\NormalTok{]}

\NormalTok{A\_Binv }\OperatorTok{=}\NormalTok{ np.linalg.inv(A[:, B])}

\CommentTok{\# A and c {-} basic}
\NormalTok{T[:m, B] }\OperatorTok{=}\NormalTok{ np.identity(m)}
\NormalTok{T[}\OperatorTok{{-}}\DecValTok{1}\NormalTok{, B] }\OperatorTok{=}\NormalTok{ np.zeros(m)}

\CommentTok{\# A and c {-} nonbasic}
\NormalTok{T[:m, N] }\OperatorTok{=}\NormalTok{ A\_Binv }\OperatorTok{@}\NormalTok{ A[:, N]}
\NormalTok{T[}\OperatorTok{{-}}\DecValTok{1}\NormalTok{, N] }\OperatorTok{=}\NormalTok{ c[N] }\OperatorTok{{-}}\NormalTok{ c[B] }\OperatorTok{@}\NormalTok{ T[:m, N]}

\CommentTok{\# b and d}
\NormalTok{T[:m, }\OperatorTok{{-}}\DecValTok{1}\NormalTok{] }\OperatorTok{=}\NormalTok{ A\_Binv }\OperatorTok{@}\NormalTok{ b}
\NormalTok{T[}\OperatorTok{{-}}\DecValTok{1}\NormalTok{, }\OperatorTok{{-}}\DecValTok{1}\NormalTok{] }\OperatorTok{=} \OperatorTok{{-}}\NormalTok{ c[B] }\OperatorTok{@}\NormalTok{ (T[:m, }\OperatorTok{{-}}\DecValTok{1}\NormalTok{])}

\CommentTok{\# make it pretty again}
\NormalTok{T }\OperatorTok{=}\NormalTok{ np.array(T, dtype }\OperatorTok{=} \BuiltInTok{object}\NormalTok{)}
\ControlFlowTok{for}\NormalTok{ j }\KeywordTok{in} \BuiltInTok{range}\NormalTok{(m }\OperatorTok{+} \DecValTok{1}\NormalTok{):}
    \ControlFlowTok{for}\NormalTok{ i }\KeywordTok{in} \BuiltInTok{range}\NormalTok{(n }\OperatorTok{+}\NormalTok{ m }\OperatorTok{+} \DecValTok{2}\NormalTok{):}
\NormalTok{        T[j][i] }\OperatorTok{=}\NormalTok{ Fraction(}\BuiltInTok{round}\NormalTok{(T[j][i]))}

\NormalTok{tableau(T)}
\end{Highlighting}
\end{Shaded}

\begin{verbatim}
|-------+-------+-------+-------+-------+-------+-------+-------+-------+
|    x1 |    x2 |    x3 |    x4 |    x5 |    x6 |    x7 |    -z |     b |
|-------+-------+-------+-------+-------+-------+-------+-------+-------+
|     1 |     1 |     0 |     0 |     1 |     2 |    -1 |     0 |    39 |
|     0 |     1 |     1 |     0 |     0 |     4 |    -1 |     0 |    48 |
|     0 |    -1 |     0 |     1 |    -1 |    -5 |     2 |     0 |    30 |
|-------+-------+-------+-------+-------+-------+-------+-------+-------+
|     0 |    -1 |     0 |     0 |    -2 |    -1 |    -3 |     1 | -1827 |
|-------+-------+-------+-------+-------+-------+-------+-------+-------+
\end{verbatim}

We achieved the exact same tableau with non-positive reduced cost
therefore we found the optimal solution - again.

\newpage

\section{3 - Reduced Cost}\label{reduced-cost}

What's the increase in price that would make \(x_2\) worth being
produced? \newline We remember that:

\[
\bar{c}_2 = c_2 - \sum_{i}{y_i a_{i2}}
\]

where \(\bar{c}_2\) is the reduced cost of \(x_2\) after last iteration.
\newline To make \(x_2\) worth producing the reduced cost should be
positive, and from the above formula we see that the increase in price
will correspond to the same increase in the reduced cost. As the reduced
cost of \(x_2\) is equal to \(-1\) after the last iteration, the
increase in price that would make \(x_2\) worth being produced is any
amount \(> 1\).

\newpage

\section{4 - Shadow Price}\label{shadow-price}

Shadow prices can be read directly from the last tableau:

\[
\begin{aligned}
2 \ & - for \ an \ hour \ of \ work \\
1 \ & - for \ a \ unit \ of \ metal \\
3 \ & - for \ a \ unit \ of \ wood
\end{aligned}
\]

\newpage

\section{5 - Complementary Slackness
Theorem}\label{complementary-slackness-theorem}

Are all the constraints binding? \newline The Complementary Slackness
Theorem implies for \(x*\) and \(y*\) - optimal solutions of the Primal
and Dual problems respectively:

\[
\forall_{j}{\ (b_j - \sum_{i}{a_{ji}x*_i})y*_j = 0}
\]

Which means that for each constraint it's either binding, or it's shadow
price is \(0\). Seeing that all of the shadow prices are non-zero
(previous subtask) we can conclude that all constraint of the Primal
problem are binding (active). \newline On a side note: We can also
observe that none of the slack variables are in basis in the optimal
solution, which means there is no slack capacity, which is yet another
way of concluding the above.

\newpage

\section{6 - Economical Interpretation}\label{economical-interpretation}

The imputed (reduced) cost of \(x_2\) is:

\[
\bar{c}_2 = c_2 - \sum_{i}{y*_i a_{i2}} = -1 < 0
\]

Which means that the cost of resources needed to produce a unit of
product \(x_2\) is significantly higher than the price of the product
itself, making it not a viable option to produce. \newline In other
words: there is slack in the \(2^{nd}\) constraint of the Dual.

\newpage

\section{7 - Sensitivity analysis - increase desk net
profit}\label{sensitivity-analysis---increase-desk-net-profit}

\subsubsection{TODO}\label{todo}




\end{document}
